\section{A Tractable Special Case}

This appendix describes a restricted setting in which the diagnostic for unfaithful cuts
is tractable. The purpose is not to provide a general algorithm, but to demonstrate that
the diagnostic reduces to a concrete, checkable condition under standard assumptions.
This addresses the question of whether the unfaithful cut is purely conceptual, by
showing that it reduces to a concrete, checkable condition in restricted regimes where
the full system dynamics are known. This does not contradict the general impossibility
result established in the main text, but rather delineates the boundary between
epistemically inaccessible and tractable cases.

\subsection{Setting}

Consider a system with a finite underlying state space $\mathcal{X}$ and a finite action
set $\mathcal{A}$. The system evolves deterministically according to a transition
function
\[
f : \mathcal{X} \times \mathcal{A} \rightarrow \mathcal{X}.
\]

A trajectory is a finite or infinite sequence
\[
h = (x_0, a_0, x_1, a_1, \ldots),
\]
generated by repeated application of $f$. We assume full knowledge of $f$ for the
purposes of this analysis.

A model interface is defined by a reduction
\[
\pi : \mathcal{X} \rightarrow \mathcal{S},
\]
which maps underlying states to abstract representational states. This induces a
reduction on trajectories by applying $\pi$ pointwise to the state components.

The set of admissible interventions consists of single-step actions:
\[
\mathcal{I} = \{ I_a : a \in \mathcal{A} \},
\]
where each intervention $I_a$ replaces the next action in a trajectory with action $a$.

\subsection{Interface Equivalence}

Two trajectories are equivalent at the interface if their underlying states map to the
same abstract state:
\[
x_1 \sim_\pi x_2 \quad \text{if and only if} \quad \pi(x_1) = \pi(x_2).
\]

Under observational evaluation, the model conditions only on $\pi(x_t)$. If predictions
depend solely on the abstract state, then interface-equivalent states are
observationally indistinguishable.

\subsection{Checking Intervention Closure}

In this setting, intervention closure reduces to a simple condition.

The reduction $\pi$ is closed under intervention if and only if, for all abstract states
$s \in \mathcal{S}$, all actions $a \in \mathcal{A}$, and all pairs of underlying states
$x_1, x_2 \in \mathcal{X}$ such that
\[
\pi(x_1) = \pi(x_2) = s,
\]
we have
\[
\pi(f(x_1, a)) = \pi(f(x_2, a)).
\]

If this condition holds, then no admissible intervention separates trajectories that are
equivalent at the interface. The cut induced by $\pi$ is faithful.

If the condition fails for any $(s, a)$, then there exists an admissible intervention
that separates interface-equivalent trajectories. The reduction is therefore an
unfaithful cut.

This condition is decidable in the finite deterministic case by exhaustive enumeration
of states and actions. No learning, optimization, or simulation beyond the known
transition function is required.

\subsection{Interpretation}

This special case illustrates several points relevant to the general diagnostic.

First, it shows that the unfaithful cut is not merely a philosophical construct: in
restricted settings with finite state spaces and known dynamics, it corresponds to a
precise closure condition that can be checked directly. In
restricted but practically relevant settings, it corresponds to a concrete violation of
closure conditions that can be checked directly.

Second, the condition depends only on the interaction between the reduction $\pi$ and
the action-dependent dynamics of the system. It is independent of how the reduction was
learned or why it was chosen.

Finally, the example clarifies why the diagnostic becomes difficult or undecidable in
more general settings. When state spaces are large or continuous, dynamics are
stochastic, or admissible interventions are complex, the exhaustive check above is no
longer tractable. The unfaithful cut persists as a structural validity condition, but its
evaluation may require domain-specific reasoning rather than computation.

Importantly, the tractability of this special case depends on assumptions that are rarely
satisfied in deployed learning systems: finite state spaces, deterministic dynamics, and
full knowledge of the transition function. When these assumptions fail—as they do in
most real-world settings—observational evaluation alone cannot certify intervention
closure. In such cases, the unfaithful cut marks a fundamental epistemic limitation
rather than a computational one.

This appendix therefore demonstrates that the diagnostic admits concrete instantiations
in special cases, while remaining a general structural criterion in the broader setting
considered in the main text.

The special case considered here is deterministic for clarity. In stochastic systems,
interventions may separate trajectories in distribution rather than deterministically.
In such settings, separation may be defined in terms of differing outcome
distributions rather than pointwise divergence. The unfaithful cut framework applies whenever interventions induce distinguishable
distributions over interface states, though determining such separation from
observational data alone is generally impossible without additional assumptions. We
leave systematic treatment of stochastic separation and approximate notions of
faithfulness to future work.