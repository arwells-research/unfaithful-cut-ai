\section{The Unfaithful Cut}

This section introduces the central diagnostic of the paper. The diagnostic is intentionally representational and epistemic in scope: it characterizes
when intervention failure is fundamentally undetectable from observational evaluation
alone, rather than proposing a new causal criterion or learning algorithm.

Any predictive model operates across an interface that defines what the model can
observe, store, and condition on. This interface induces a reduction from underlying
system trajectories to a smaller set of representational states. In practice, many
distinct trajectories may be mapped to the same interface representation.

Such reductions are unavoidable and are not inherently problematic. Indeed, they are
necessary for learning and generalization. The issue addressed here is not reduction
itself, but a failure mode in which a reduction preserves predictive accuracy while
destroying distinctions required for intervention.

We refer to this failure as an \emph{Unfaithful Cut}.

A reduction is said to be \emph{closed under prediction} if all trajectories that are
equivalent at the model interface yield identical predictions for all observational
queries evaluated during training and validation. This notion corresponds to observational adequacy: the model reproduces the statistics of observed data at its interface.

A reduction is said to be \emph{closed under intervention} if all trajectories that are
equivalent at the model interface remain equivalent under all admissible interventions.
This notion corresponds to interventional validity: the interface preserves all
distinctions that actions can later make operationally relevant.

Concrete illustrations of this epistemic limitation are provided in Appendices A–C.

\paragraph{Diagnostic Definition (Unfaithful Cut).}
An \emph{Unfaithful Cut} is a reduction that is closed under prediction but not closed
under intervention. Equivalently, an unfaithful cut exists when two trajectories are
indistinguishable at the model interface under observation, yet are separated by at
least one admissible intervention.

This failure is typically invisible during training. Standard learning procedures
expose the model only to observational queries drawn from a fixed data-generating
process. If two trajectories are observationally indistinguishable at the interface,
no predictive objective can force the model to distinguish them. The reduction
therefore appears sufficient by all standard metrics.

The failure manifests only at deployment, when interventions are introduced. Actions
act on the underlying system, not on the reduced representation. As a result,
interventions can expose distinctions between trajectories that were erased by the
cut. When this occurs, the model encounters a regime in which its interface no longer
supports reliable control.

This failure is not attributable to data scarcity or optimization error.
Rather, it arises because the representation induced by the model interface is not
intervention-consistent. The limitation is structural and epistemic: it concerns what
cannot be inferred from observation alone.

\subsection{Formal Definitions}

We now provide a minimal formalization sufficient to make the notion of an unfaithful cut
precise. The goal is not to impose a particular modeling framework, but to fix notation
and clarify the structural content of the definitions used throughout the paper.

Let $\mathcal{H}$ denote the set of admissible trajectories of a system. A trajectory
$h \in \mathcal{H}$ represents a complete system history, including all variables and
degrees of freedom relevant to the underlying dynamics. No assumptions are made about
how $\mathcal{H}$ is generated or whether it is known explicitly. This abstraction is intentional: the diagnostic concerns representational sufficiency
under admissible interventions, not the generative mechanisms or stochastic structure of the underlying process.

A predictive model operates through an interface that induces a reduction of
$\mathcal{H}$ to a representational space $\mathcal{S}$. We represent this reduction by
a mapping
\[
\pi : \mathcal{H} \rightarrow \mathcal{S},
\]
which assigns each trajectory to a model state or interface representation.

This mapping induces an equivalence relation on trajectories:
\[
h_1 \sim_\pi h_2 \quad \text{if and only if} \quad \pi(h_1) = \pi(h_2).
\]
Trajectories equivalent under $\sim_\pi$ are indistinguishable at the model interface.

We represent interventions abstractly as operators acting on trajectories. Let
$\mathcal{I}$ denote a set of admissible interventions, where each
$I \in \mathcal{I}$ maps trajectories to trajectories:
\[
I : \mathcal{H} \rightarrow \mathcal{H}.
\]
An intervention may represent an action, control input, or policy change, understood as
producing a counterfactual trajectory corresponding to applying that operation at the
interface. No assumptions are made about the internal structure of interventions beyond
their effect on trajectories.

Admissible interventions are defined relative to a deployment context and intended use
of the model. Informally, an intervention is admissible if it corresponds to an action
or manipulation that the model may be used to evaluate or support in practice. The
diagnostic depends only on the existence of at least one admissible intervention that
separates interface-equivalent trajectories, not on exhaustive enumeration of all
possible interventions.

\paragraph{Prediction closure.}
The reduction $\pi$ is said to be \emph{closed under prediction} if, for all
$h_1, h_2 \in \mathcal{H}$ such that $h_1 \sim_\pi h_2$, the model assigns identical
predictions to $h_1$ and $h_2$ for all observational queries evaluated during training
and validation. This corresponds to observational adequacy.

\paragraph{Intervention closure.}
The reduction $\pi$ is said to be \emph{closed under intervention} if, for all
$h_1, h_2 \in \mathcal{H}$ such that $h_1 \sim_\pi h_2$, and for all admissible
interventions $I \in \mathcal{I}$,
\[
I(h_1) \sim_\pi I(h_2).
\]
That is, trajectories that are indistinguishable at the interface remain
indistinguishable after any admissible intervention.

\paragraph{Unfaithful Cut.}
An \emph{Unfaithful Cut} occurs when the reduction $\pi$ is closed under prediction but
not closed under intervention. Equivalently, there exist trajectories
$h_1, h_2 \in \mathcal{H}$ and an admissible intervention $I \in \mathcal{I}$ such that
$h_1 \sim_\pi h_2$ but $I(h_1) \not\sim_\pi I(h_2)$.

This formalization makes explicit that the unfaithful cut is a structural property of the
representation induced by the model interface. It does not depend on learning
procedures, data availability, or optimization quality, and can arise even in
deterministic systems with perfect observational prediction.

Trajectories that are indistinguishable at the interface under observation may
nonetheless diverge when acted upon by admissible interventions.

\paragraph{Scope and relation to causal insufficiency.}
An unfaithful cut corresponds, from a causal inference perspective, to a form of causal
insufficiency: histories that differ in their response to intervention are collapsed by
the interface representation. The contribution of the unfaithful cut is not to identify
a failure mode distinct from causal insufficiency, but to make explicit how such failures
arise from representational interface choices, through representational interface choices that determine whether causal distinctions are preserved or erased.

Framed this way, the unfaithful cut highlights a fundamental limitation of observational
evaluation: even with perfect prediction over the interface variables, intervention
behavior may remain indeterminate when histories with distinct intervention responses are
collapsed. This limitation is epistemic rather than statistical, and cannot be resolved
by additional data alone.

The unfaithful cut does not identify failures beyond those recognizable as causal
insufficiency; rather, it provides a representational perspective that makes explicit how
such insufficiency can arise from interface design choices, particularly in learned
systems where causal structure is implicit rather than specified.
