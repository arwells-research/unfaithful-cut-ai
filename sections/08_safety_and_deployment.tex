\section{Implications for AI Safety and Deployment}

Many failures attributed to deployment risk are not caused by instability,
overfitting, or insufficient optimization. This section focuses on representational risk rather than deployment workflows, which
are necessarily domain-specific. Instead, they arise when a model is
applied in a regime that requires distinctions its representation cannot support.
In such cases, predictive performance is a poor proxy for reliability under
intervention.

The Unfaithful Cut clarifies why standard safety checks can be insufficient.
Evaluation protocols typically probe a model under observation or under limited
action distributions. If the model interface collapses trajectories that are only
distinguishable under intervention, these checks cannot reveal the resulting
failure mode. A model may therefore appear stable and well-calibrated while
remaining incapable of supporting reliable control.

From a deployment perspective, reliability depends not only on how well a model
predicts, but on whether its representation is intervention-consistent for the
actions it will be asked to support. This consideration is orthogonal to training
quality: a well-trained model can still be unsafe under intervention if its
interface induces an unfaithful cut.

The diagnostic presented in this paper enables a complementary form of
pre-deployment assessment. Rather than asking whether a model performs well on
held-out data, it asks whether the model’s interface preserves the distinctions
that admissible interventions can make operationally relevant. When the answer is
negative, additional data or optimization cannot guarantee reliable behavior,
because the failure lies in the representation itself.

For researchers, this perspective emphasizes auditing model interfaces in addition to
optimizing predictive metrics; for practitioners, it cautions against relying on
observational validation when models are used to guide intervention.

The implication is not that intervention should be avoided, but that
representational assumptions must be audited before intervention is attempted.
Identifying unfaithful cuts provides a principled way to delineate the conditions
under which predictive models can be relied upon for decision-making and control.