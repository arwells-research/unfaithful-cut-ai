\section{Conclusion}

Predictive accuracy and interventional reliability are distinct properties of a model,
and conflating them obscures a class of structural failure modes that remain invisible
under standard evaluation.
A system may reproduce observed outcomes with high fidelity and still be incapable of
supporting reliable intervention. This failure arises when a representational reduction
collapses distinct trajectories into a single interface state, preserving observational
adequacy while breaking the counterfactual distinctions required for control.

We have formalized this failure as an \emph{Unfaithful Cut}. It is not a consequence of
insufficient data, poor optimization, or limited model capacity, but a limitation induced
by representational interface choices. Because the failure is structural and epistemic,
it cannot be repaired by additional training or improved predictive performance alone.

The contribution of this work is diagnostic rather than constructive. We have presented a
criterion for identifying when a predictive model cannot, in principle, be trusted under
intervention, and illustrated its relevance in sim-to-real robotics, offline
reinforcement learning, and causal inference. While the diagnostic does not prescribe how
to construct faithful representations, it delineates a boundary: beyond this boundary,
interventional behavior is fundamentally underdetermined by observational data.

This perspective suggests directions for future research. Faithful representations are
guaranteed when the interface preserves full intervention-sufficient state, and may be
approximated through domain-informed state augmentation guided by known intervention
effects. More generally, active intervention during data collection, formal verification
in restricted domains, and information-theoretic characterizations of
intervention-consistent reductions are promising avenues for developing constructive
methods grounded in the diagnostic framework introduced here.

As predictive systems increasingly mediate real-world decisions, distinguishing
observational adequacy from interventional reliability is essential for responsible
deployment. The unfaithful cut makes explicit a fundamental limit of observational
evaluation: some intervention failures are undetectable from observation alone, regardless of predictive accuracy.