\section{Illustrative Toy Example: An Unfaithful Cut}

This appendix provides a minimal, concrete example illustrating an unfaithful cut and
the diagnostic described in Section~6. The example is intentionally simple and does not
involve learning, estimation, or function approximation. Its purpose is not to introduce
a new causal pathology, but to make explicit a fundamental epistemic limitation of
observational evaluation.

Specifically, the example shows how two system histories that are indistinguishable at a
chosen model interface can nonetheless diverge under admissible intervention, even when
observational prediction at that interface is perfect. From a causal perspective, this
corresponds to a form of causal insufficiency induced by representational choice. The
unfaithful cut framework makes this failure explicit at the level of the interface,
independent of any particular causal formalism.

\subsection{The System}

Consider a deterministic system with an internal mode variable
$m \in \{0,1\}$, an observable variable $x \in \mathbb{R}$, and a binary action
$a \in \{0,1\}$. The internal mode $m$ is not observable at the model interface.

The system evolves as follows. Under passive observation (no intervention),
both modes produce identical observations:
\[
x_t = 0 \quad \text{for all } t,
\]
regardless of the value of $m$.

Thus, under observation, trajectories corresponding to $m=0$ and $m=1$ are
indistinguishable.

\subsection{The Reduction}

Suppose a model is constructed whose interface consists solely of the observable
variable $x$. The model therefore represents system state using $x_t$ and ignores
the internal mode $m$.

This induces a reduction in which all trajectories with different values of $m$
are mapped to the same interface representation whenever $x_t = 0$.

Under observational evaluation, this reduction is sufficient. The model can
predict observed outcomes perfectly, since all observed data are reproduced
exactly at the interface.

The reduction is therefore closed under prediction.

\subsection{The Intervention Failure}

Now consider an admissible intervention defined by taking action $a=1$ at time
$t$.

The effect of the intervention depends on the internal mode:
\[
\text{if } m = 0, \quad x_{t+1} = +1,
\]
\[
\text{if } m = 1, \quad x_{t+1} = -1.
\]

Thus, the same action separates trajectories that were previously indistinguishable
at the model interface. Two trajectories with identical interface state $x_t = 0$
lead to different outcomes under intervention.

Because the model cannot distinguish the underlying modes, it must assign the
same predicted outcome to both cases. Under intervention, this prediction is
necessarily incorrect for at least one trajectory.

This constitutes an intervention failure despite perfect observational accuracy.

\subsection{The Diagnostic}

We now apply the diagnostic from Section~6.

First, identify the model interface. Here, the interface consists solely of the
observable variable $x$.

Second, identify interface equivalence classes. All trajectories with different
values of $m$ are equivalent at the interface whenever $x_t = 0$.

Third, identify admissible interventions. The action $a=1$ is admissible at time
$t$.

Finally, ask whether the intervention separates trajectories within an
interface-equivalence class. In this system, the action $a=1$ leads to different
outcomes depending on the value of $m$, despite identical interface states prior
to intervention.

The diagnostic therefore flags an unfaithful cut. The reduction is closed under
prediction but not closed under intervention.

This example demonstrates that intervention failure can arise purely from
representational structure, independent of data, learning, or optimization. Crucially,
the failure is not detectable from observational data alone: no amount of passive
evaluation at the interface can distinguish the histories or anticipate their divergent
responses to intervention. The unfaithful cut makes explicit this epistemic limitation
and shows how it follows directly from the choice of interface representation.